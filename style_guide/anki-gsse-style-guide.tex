% Created 2020-11-29 Sun 12:19
% Intended LaTeX compiler: pdflatex
\documentclass[11pt]{article}
\usepackage[utf8]{inputenc}
\usepackage[T1]{fontenc}
\usepackage{graphicx}
\usepackage{grffile}
\usepackage{longtable}
\usepackage{wrapfig}
\usepackage{rotating}
\usepackage[normalem]{ulem}
\usepackage{amsmath}
\usepackage{textcomp}
\usepackage{amssymb}
\usepackage{capt-of}
\usepackage{hyperref}
\author{Christopher Chen}
\date{\today}
\title{Anki for GSSE style guide by example}
\hypersetup{
 pdfauthor={Christopher Chen},
 pdftitle={Anki for GSSE style guide by example},
 pdfkeywords={},
 pdfsubject={},
 pdfcreator={Emacs 27.1 (Org mode 9.4)}, 
 pdflang={English}}
\begin{document}

\maketitle
\tableofcontents


\section{Recap of \href{https://www.supermemo.com/articles/20rules.htm}{SuperMemo's 20 rules} (summarised)}
\label{sec:org408fb5e}
\begin{enumerate}
\item Do not learn if you do not understand
\item Learn before you memorise: build an overall picture before you make cards
\item Build upon the basics: do not neglect the seemingly easy things
\item Keep the cards simple: split information into small digestible chunks
\item Cloze deletions are easy and effective
\item Use visual stimuli to boost retention
\item Use mnemonic techniques when possible
\item Graphic deletion (image occlusion) is as good as cloze deletion
\item Avoid big sets and long lists: break them down into smaller groups if possible
\item Avoid enumerations (e.g. having a whole poem on one card): try using smaller overlapped cloze deletions
\item Fight to distinguish similar pieces of information from one another and avoid confusion
\item Optimise wording so that there are fewer words and bigger impact
\item Refer to other memories (perhaps those that you built in the deck earlier)
\item Personalise and give examples (to an extent because it is a public deck after all)
\item Use vivid and emotional examples (humour is particularly useful)
\item Give the context of the answer as a hint to simplify wording
\item Redundancy can sometimes be useful and does not break rule 4 e.g. multiple solutions to a problem, multiple explanations of a concept
\item Provide sources for your knowledge if possible and relevant (applies esp to anatomy, lots of contradicting information between different textbooks)
\item Date-stamp knowledge that can be relatively volatile (not as applicable to GSSE since Last's 9e has been in use for ages, but still applicable to e.g. anatomical variants, new scientific findings)
\item Prioritise knowledge that is high yield
\end{enumerate}
\section{Examples of good and bad cards from Mickey's deck}
\label{sec:org1affcc1}
\subsection{Rule 3: build upon the basics}
\label{sec:org707819a}
Don't neglect the stuff that seems simple. This is an example of a good card:
\subsection{Rule 4: keep it simple, stupid!}
\label{sec:org853b459}
Anki (and spaced repetition, and human memory in general) works best with information that is \emph{atomised}. That is, if you think you can split it into smaller pieces, \textbf{do it}, unless it makes more sense to learn something as a relationship (e.g. a pair, an equation, a rule of thumb, etc.).

PLEASE, PLEASE, PLEASE DO NOT EVER DO SOMETHING LIKE THIS. THIS IS \textbf{BAD}.
Lengthy, wordy paragraphs do not stick. If you forget 30\% of the paragraph, you will have failed the whole card. It makes much more sense to split it into smaller pieces so that you can more easily target the specific aspects you are forgetting.

This kind of information can also be considered an enumeration since it follows a logical sequence. So structure the cards in such a way that they follow a sequence, e.g.:

\subsection{Rule 6: use more visuals}
\label{sec:orgb4dd9cb}
\subsection{Rule 7: mnemonic techniques}
\label{sec:org3b3bc33}
\subsection{Rule 9: sets and lists}
\label{sec:org475c38b}
\subsection{Rule 10: enumerations}
\label{sec:org089caa0}
\end{document}
